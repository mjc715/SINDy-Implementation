% Options for packages loaded elsewhere
\PassOptionsToPackage{unicode}{hyperref}
\PassOptionsToPackage{hyphens}{url}
\PassOptionsToPackage{dvipsnames,svgnames,x11names}{xcolor}
%
\documentclass[
]{article}

\usepackage{amsmath,amssymb}
\usepackage{iftex}
\ifPDFTeX
  \usepackage[T1]{fontenc}
  \usepackage[utf8]{inputenc}
  \usepackage{textcomp} % provide euro and other symbols
\else % if luatex or xetex
  \usepackage{unicode-math}
  \defaultfontfeatures{Scale=MatchLowercase}
  \defaultfontfeatures[\rmfamily]{Ligatures=TeX,Scale=1}
\fi
\usepackage{lmodern}
\ifPDFTeX\else  
    % xetex/luatex font selection
\fi
% Use upquote if available, for straight quotes in verbatim environments
\IfFileExists{upquote.sty}{\usepackage{upquote}}{}
\IfFileExists{microtype.sty}{% use microtype if available
  \usepackage[]{microtype}
  \UseMicrotypeSet[protrusion]{basicmath} % disable protrusion for tt fonts
}{}
\makeatletter
\@ifundefined{KOMAClassName}{% if non-KOMA class
  \IfFileExists{parskip.sty}{%
    \usepackage{parskip}
  }{% else
    \setlength{\parindent}{0pt}
    \setlength{\parskip}{6pt plus 2pt minus 1pt}}
}{% if KOMA class
  \KOMAoptions{parskip=half}}
\makeatother
\usepackage{xcolor}
\setlength{\emergencystretch}{3em} % prevent overfull lines
\setcounter{secnumdepth}{5}
% Make \paragraph and \subparagraph free-standing
\makeatletter
\ifx\paragraph\undefined\else
  \let\oldparagraph\paragraph
  \renewcommand{\paragraph}{
    \@ifstar
      \xxxParagraphStar
      \xxxParagraphNoStar
  }
  \newcommand{\xxxParagraphStar}[1]{\oldparagraph*{#1}\mbox{}}
  \newcommand{\xxxParagraphNoStar}[1]{\oldparagraph{#1}\mbox{}}
\fi
\ifx\subparagraph\undefined\else
  \let\oldsubparagraph\subparagraph
  \renewcommand{\subparagraph}{
    \@ifstar
      \xxxSubParagraphStar
      \xxxSubParagraphNoStar
  }
  \newcommand{\xxxSubParagraphStar}[1]{\oldsubparagraph*{#1}\mbox{}}
  \newcommand{\xxxSubParagraphNoStar}[1]{\oldsubparagraph{#1}\mbox{}}
\fi
\makeatother

\usepackage{color}
\usepackage{fancyvrb}
\newcommand{\VerbBar}{|}
\newcommand{\VERB}{\Verb[commandchars=\\\{\}]}
\DefineVerbatimEnvironment{Highlighting}{Verbatim}{commandchars=\\\{\}}
% Add ',fontsize=\small' for more characters per line
\usepackage{framed}
\definecolor{shadecolor}{RGB}{241,243,245}
\newenvironment{Shaded}{\begin{snugshade}}{\end{snugshade}}
\newcommand{\AlertTok}[1]{\textcolor[rgb]{0.68,0.00,0.00}{#1}}
\newcommand{\AnnotationTok}[1]{\textcolor[rgb]{0.37,0.37,0.37}{#1}}
\newcommand{\AttributeTok}[1]{\textcolor[rgb]{0.40,0.45,0.13}{#1}}
\newcommand{\BaseNTok}[1]{\textcolor[rgb]{0.68,0.00,0.00}{#1}}
\newcommand{\BuiltInTok}[1]{\textcolor[rgb]{0.00,0.23,0.31}{#1}}
\newcommand{\CharTok}[1]{\textcolor[rgb]{0.13,0.47,0.30}{#1}}
\newcommand{\CommentTok}[1]{\textcolor[rgb]{0.37,0.37,0.37}{#1}}
\newcommand{\CommentVarTok}[1]{\textcolor[rgb]{0.37,0.37,0.37}{\textit{#1}}}
\newcommand{\ConstantTok}[1]{\textcolor[rgb]{0.56,0.35,0.01}{#1}}
\newcommand{\ControlFlowTok}[1]{\textcolor[rgb]{0.00,0.23,0.31}{\textbf{#1}}}
\newcommand{\DataTypeTok}[1]{\textcolor[rgb]{0.68,0.00,0.00}{#1}}
\newcommand{\DecValTok}[1]{\textcolor[rgb]{0.68,0.00,0.00}{#1}}
\newcommand{\DocumentationTok}[1]{\textcolor[rgb]{0.37,0.37,0.37}{\textit{#1}}}
\newcommand{\ErrorTok}[1]{\textcolor[rgb]{0.68,0.00,0.00}{#1}}
\newcommand{\ExtensionTok}[1]{\textcolor[rgb]{0.00,0.23,0.31}{#1}}
\newcommand{\FloatTok}[1]{\textcolor[rgb]{0.68,0.00,0.00}{#1}}
\newcommand{\FunctionTok}[1]{\textcolor[rgb]{0.28,0.35,0.67}{#1}}
\newcommand{\ImportTok}[1]{\textcolor[rgb]{0.00,0.46,0.62}{#1}}
\newcommand{\InformationTok}[1]{\textcolor[rgb]{0.37,0.37,0.37}{#1}}
\newcommand{\KeywordTok}[1]{\textcolor[rgb]{0.00,0.23,0.31}{\textbf{#1}}}
\newcommand{\NormalTok}[1]{\textcolor[rgb]{0.00,0.23,0.31}{#1}}
\newcommand{\OperatorTok}[1]{\textcolor[rgb]{0.37,0.37,0.37}{#1}}
\newcommand{\OtherTok}[1]{\textcolor[rgb]{0.00,0.23,0.31}{#1}}
\newcommand{\PreprocessorTok}[1]{\textcolor[rgb]{0.68,0.00,0.00}{#1}}
\newcommand{\RegionMarkerTok}[1]{\textcolor[rgb]{0.00,0.23,0.31}{#1}}
\newcommand{\SpecialCharTok}[1]{\textcolor[rgb]{0.37,0.37,0.37}{#1}}
\newcommand{\SpecialStringTok}[1]{\textcolor[rgb]{0.13,0.47,0.30}{#1}}
\newcommand{\StringTok}[1]{\textcolor[rgb]{0.13,0.47,0.30}{#1}}
\newcommand{\VariableTok}[1]{\textcolor[rgb]{0.07,0.07,0.07}{#1}}
\newcommand{\VerbatimStringTok}[1]{\textcolor[rgb]{0.13,0.47,0.30}{#1}}
\newcommand{\WarningTok}[1]{\textcolor[rgb]{0.37,0.37,0.37}{\textit{#1}}}

\providecommand{\tightlist}{%
  \setlength{\itemsep}{0pt}\setlength{\parskip}{0pt}}\usepackage{longtable,booktabs,array}
\usepackage{calc} % for calculating minipage widths
% Correct order of tables after \paragraph or \subparagraph
\usepackage{etoolbox}
\makeatletter
\patchcmd\longtable{\par}{\if@noskipsec\mbox{}\fi\par}{}{}
\makeatother
% Allow footnotes in longtable head/foot
\IfFileExists{footnotehyper.sty}{\usepackage{footnotehyper}}{\usepackage{footnote}}
\makesavenoteenv{longtable}
\usepackage{graphicx}
\makeatletter
\def\maxwidth{\ifdim\Gin@nat@width>\linewidth\linewidth\else\Gin@nat@width\fi}
\def\maxheight{\ifdim\Gin@nat@height>\textheight\textheight\else\Gin@nat@height\fi}
\makeatother
% Scale images if necessary, so that they will not overflow the page
% margins by default, and it is still possible to overwrite the defaults
% using explicit options in \includegraphics[width, height, ...]{}
\setkeys{Gin}{width=\maxwidth,height=\maxheight,keepaspectratio}
% Set default figure placement to htbp
\makeatletter
\def\fps@figure{htbp}
\makeatother

\usepackage{algorithm}
\usepackage{algpseudocode}
\usepackage[left=2cm,right=2cm,top = 2cm,bottom = 2cm]{geometry}
\usepackage[style=alphabetic]{biblatex}
\makeatletter
\@ifpackageloaded{tcolorbox}{}{\usepackage[skins,breakable]{tcolorbox}}
\@ifpackageloaded{fontawesome5}{}{\usepackage{fontawesome5}}
\definecolor{quarto-callout-color}{HTML}{909090}
\definecolor{quarto-callout-note-color}{HTML}{0758E5}
\definecolor{quarto-callout-important-color}{HTML}{CC1914}
\definecolor{quarto-callout-warning-color}{HTML}{EB9113}
\definecolor{quarto-callout-tip-color}{HTML}{00A047}
\definecolor{quarto-callout-caution-color}{HTML}{FC5300}
\definecolor{quarto-callout-color-frame}{HTML}{acacac}
\definecolor{quarto-callout-note-color-frame}{HTML}{4582ec}
\definecolor{quarto-callout-important-color-frame}{HTML}{d9534f}
\definecolor{quarto-callout-warning-color-frame}{HTML}{f0ad4e}
\definecolor{quarto-callout-tip-color-frame}{HTML}{02b875}
\definecolor{quarto-callout-caution-color-frame}{HTML}{fd7e14}
\makeatother
\makeatletter
\@ifpackageloaded{caption}{}{\usepackage{caption}}
\AtBeginDocument{%
\ifdefined\contentsname
  \renewcommand*\contentsname{Table of contents}
\else
  \newcommand\contentsname{Table of contents}
\fi
\ifdefined\listfigurename
  \renewcommand*\listfigurename{List of Figures}
\else
  \newcommand\listfigurename{List of Figures}
\fi
\ifdefined\listtablename
  \renewcommand*\listtablename{List of Tables}
\else
  \newcommand\listtablename{List of Tables}
\fi
\ifdefined\figurename
  \renewcommand*\figurename{Figure}
\else
  \newcommand\figurename{Figure}
\fi
\ifdefined\tablename
  \renewcommand*\tablename{Table}
\else
  \newcommand\tablename{Table}
\fi
}
\@ifpackageloaded{float}{}{\usepackage{float}}
\floatstyle{ruled}
\@ifundefined{c@chapter}{\newfloat{codelisting}{h}{lop}}{\newfloat{codelisting}{h}{lop}[chapter]}
\floatname{codelisting}{Listing}
\newcommand*\listoflistings{\listof{codelisting}{List of Listings}}
\makeatother
\makeatletter
\makeatother
\makeatletter
\@ifpackageloaded{caption}{}{\usepackage{caption}}
\@ifpackageloaded{subcaption}{}{\usepackage{subcaption}}
\makeatother
\makeatletter
\@ifpackageloaded{tikz}{}{\usepackage{tikz}}
\makeatother
        \newcommand*\circled[1]{\tikz[baseline=(char.base)]{
          \node[shape=circle,draw,inner sep=1pt] (char) {{\scriptsize#1}};}}  
                  
\ifLuaTeX
  \usepackage{selnolig}  % disable illegal ligatures
\fi
\usepackage[]{biblatex}
\addbibresource{refs.bib}
\usepackage{bookmark}

\IfFileExists{xurl.sty}{\usepackage{xurl}}{} % add URL line breaks if available
\urlstyle{same} % disable monospaced font for URLs
\hypersetup{
  pdftitle={Report on SINDy progress},
  pdfauthor={Gage Bonner and Michael Castellucci},
  colorlinks=true,
  linkcolor={blue},
  filecolor={Maroon},
  citecolor={Blue},
  urlcolor={Blue},
  pdfcreator={LaTeX via pandoc}}

\title{Report on SINDy progress}
\author{Gage Bonner and Michael Castellucci}
\date{}

\begin{document}
\maketitle

\renewcommand*\contentsname{Table of contents}
{
\hypersetup{linkcolor=}
\setcounter{tocdepth}{3}
\tableofcontents
}
\begin{center}\rule{0.5\linewidth}{0.5pt}\end{center}

\section{Summary of work}\label{summary-of-work}

In this project, we investigated the application of \emph{data
discovery} algorithms to learn the governing equations for a dynamical
system from numerical realizations of their trajectories. We apply the
sparse identification of nonlinear dynamics (``SINDy'') as originally
proposed in \cite{brunton2016discovering} as well as a number of its
extensions. The core idea behind these algorithms is that many dynamical
systems in science are represented by differential equation systems
\(\dot{x} = f(x)\) where \(f(x)\) often a linear combination of a small
number of elementary functions. By contrast, the trajectories \(x(t)\)
may be extremely complicated and ill-suited to direct fitting. Consider
the classic Lorenz system \begin{subequations} \label{eq:lorenz-def}
\begin{align} 
    \dot{x} &= 10 (y - x), \\
    \dot{y} &= x (28 - z) - y, \\ 
    \dot{z} &= x y - (8 / 3) z .
\end{align}
\end{subequations} Trajectories \(X(t) = (x(t), y(t), z(t))\) of this
system are chaotic, making it difficult or impossible to propose a
function that fits \(X(t)\) directly. On the other hand, \(\dot{X}\) is
a low order polynomial function of \(X(t)\) and hence should be much
more tractable in principal. At the highest level, we therefore have the
following problem: given numerical trajectories
\(\{x(t_1), x(t_2), \dots \}\), compute numerically
\(\{\dot{x}(t_1), \dot{x}(t_2), \dots \}\) and attempt to find a
parsimonious (equivalently: \emph{sparse}) combination of simple
functions of the \(x(t)\) that faithfully represents it.

This report contains embedded code from the
\href{https://julialang.org/}{Julia} programming language.

\section{Core algorithms}\label{core-algorithms}

\subsection{Sparse representations}\label{sparse-representations}

Suppose that several values of a function of
\(f : \mathbb{R} \to \mathbb{R}\) are observed,
\(\{f(x_1), f(x_2), \dots \}\). A \emph{sparse representation} seeks to
represent \(f\) as a linear combination of elementary functions of \(x\)
that contains as few terms as possible while still faithfully
representing the behavior of \(f\). As a first example, take
\(f(x) = x + 2x^2 + 5x^3\). We will add a small noise term to simulate
real data.

\begin{Shaded}
\begin{Highlighting}[]
\FunctionTok{f\_simple}\NormalTok{(x) }\OperatorTok{=}\NormalTok{ x}\OperatorTok{\^{}}\FloatTok{2} \OperatorTok{+} \FloatTok{2}\NormalTok{x}\OperatorTok{\^{}}\FloatTok{2} \OperatorTok{+} \FloatTok{5}\NormalTok{x}\OperatorTok{\^{}}\FloatTok{2}
\NormalTok{x\_points }\OperatorTok{=} \FunctionTok{range}\NormalTok{(}\FloatTok{0}\NormalTok{, }\FloatTok{1}\NormalTok{, length }\OperatorTok{=} \FloatTok{10}\NormalTok{)}
\NormalTok{f\_points }\OperatorTok{=}\NormalTok{ [}\FunctionTok{f\_simple}\NormalTok{(x) }\OperatorTok{+} \FloatTok{0.1}\FunctionTok{*rand}\NormalTok{() for x }\KeywordTok{in}\NormalTok{ x\_points]}
\end{Highlighting}
\end{Shaded}

Assuming now that we

\begin{figure}
    \centering
    \includegraphics[width = 0.5\textwidth]{figures/test.png}
    \caption{test}
    \label{fig:test}
\end{figure}

\subsection{SINDy}\label{sindy}

\subsection{E-SINDy}\label{e-sindy}

\section{Julia Implementation}\label{julia-implementation}

\section{Results}\label{results}

\subsection{Direct sparse
representations}\label{direct-sparse-representations}

\subsection{Fluid}\label{fluid}

\subsection{Slow}\label{slow}

\subsection{Full}\label{full}

\section{Conclusions}\label{conclusions}

\begin{Shaded}
\begin{Highlighting}[]
\FloatTok{1} \OperatorTok{+} \FloatTok{2}
\end{Highlighting}
\end{Shaded}

\begin{verbatim}
3
\end{verbatim}

\begin{Shaded}
\begin{Highlighting}[]
\FloatTok{1} \OperatorTok{+} \FloatTok{2}
\end{Highlighting}
\end{Shaded}

\begin{verbatim}
3
\end{verbatim}

Blah blah \(sdfadf\) \begin{equation}
F = ma
\end{equation}

\begin{tcolorbox}[enhanced jigsaw, colbacktitle=quarto-callout-tip-color!10!white, colframe=quarto-callout-tip-color-frame, breakable, bottomrule=.15mm, leftrule=.75mm, arc=.35mm, coltitle=black, rightrule=.15mm, titlerule=0mm, left=2mm, opacityback=0, colback=white, bottomtitle=1mm, title={Algorithm 1 (stock)}, toprule=.15mm, toptitle=1mm, opacitybacktitle=0.6]

\phantomsection\label{annotated-cell-6}%
\begin{Shaded}
\begin{Highlighting}[]
\NormalTok{penguins }\OperatorTok{=} \FloatTok{1} \hspace*{\fill}\NormalTok{\circled{1}}
\NormalTok{penguins}\OperatorTok{*}\FloatTok{2} \hspace*{\fill}\NormalTok{\circled{2}}
\end{Highlighting}
\end{Shaded}

\begin{description}
\tightlist
\item[\circled{1}]
Take \texttt{penguins}, and then,
\item[\circled{2}]
add new columns for the bill ratio and bill area\ldots{}
\end{description}

\begin{verbatim}
2
\end{verbatim}

\end{tcolorbox}

\begin{algorithm}
\caption{An algorithm with caption}\label{alg:cap}
\begin{algorithmic}
\Require $n \geq 0$
\Ensure $y = x^n$
\State $y \gets 1$
\State $X \gets x$
\State $N \gets n$
\While{$N \neq 0$}
\If{$N$ is even}
    \State $X \gets X \times X$
    \State $N \gets \frac{N}{2}$  \Comment{This is a comment}
\ElsIf{$N$ is odd}
    \State $y \gets y \times X$
    \State $N \gets N - 1$
\EndIf
\EndWhile
\end{algorithmic}
\end{algorithm}


\printbibliography


\end{document}
