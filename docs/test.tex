% Options for packages loaded elsewhere
\PassOptionsToPackage{unicode}{hyperref}
\PassOptionsToPackage{hyphens}{url}
\PassOptionsToPackage{dvipsnames,svgnames,x11names}{xcolor}
%
\documentclass[
]{article}

\usepackage{amsmath,amssymb}
\usepackage{iftex}
\ifPDFTeX
  \usepackage[T1]{fontenc}
  \usepackage[utf8]{inputenc}
  \usepackage{textcomp} % provide euro and other symbols
\else % if luatex or xetex
  \usepackage{unicode-math}
  \defaultfontfeatures{Scale=MatchLowercase}
  \defaultfontfeatures[\rmfamily]{Ligatures=TeX,Scale=1}
\fi
\usepackage{lmodern}
\ifPDFTeX\else  
    % xetex/luatex font selection
\fi
% Use upquote if available, for straight quotes in verbatim environments
\IfFileExists{upquote.sty}{\usepackage{upquote}}{}
\IfFileExists{microtype.sty}{% use microtype if available
  \usepackage[]{microtype}
  \UseMicrotypeSet[protrusion]{basicmath} % disable protrusion for tt fonts
}{}
\makeatletter
\@ifundefined{KOMAClassName}{% if non-KOMA class
  \IfFileExists{parskip.sty}{%
    \usepackage{parskip}
  }{% else
    \setlength{\parindent}{0pt}
    \setlength{\parskip}{6pt plus 2pt minus 1pt}}
}{% if KOMA class
  \KOMAoptions{parskip=half}}
\makeatother
\usepackage{xcolor}
\setlength{\emergencystretch}{3em} % prevent overfull lines
\setcounter{secnumdepth}{5}
% Make \paragraph and \subparagraph free-standing
\makeatletter
\ifx\paragraph\undefined\else
  \let\oldparagraph\paragraph
  \renewcommand{\paragraph}{
    \@ifstar
      \xxxParagraphStar
      \xxxParagraphNoStar
  }
  \newcommand{\xxxParagraphStar}[1]{\oldparagraph*{#1}\mbox{}}
  \newcommand{\xxxParagraphNoStar}[1]{\oldparagraph{#1}\mbox{}}
\fi
\ifx\subparagraph\undefined\else
  \let\oldsubparagraph\subparagraph
  \renewcommand{\subparagraph}{
    \@ifstar
      \xxxSubParagraphStar
      \xxxSubParagraphNoStar
  }
  \newcommand{\xxxSubParagraphStar}[1]{\oldsubparagraph*{#1}\mbox{}}
  \newcommand{\xxxSubParagraphNoStar}[1]{\oldsubparagraph{#1}\mbox{}}
\fi
\makeatother

\usepackage{color}
\usepackage{fancyvrb}
\newcommand{\VerbBar}{|}
\newcommand{\VERB}{\Verb[commandchars=\\\{\}]}
\DefineVerbatimEnvironment{Highlighting}{Verbatim}{commandchars=\\\{\}}
% Add ',fontsize=\small' for more characters per line
\usepackage{framed}
\definecolor{shadecolor}{RGB}{241,243,245}
\newenvironment{Shaded}{\begin{snugshade}}{\end{snugshade}}
\newcommand{\AlertTok}[1]{\textcolor[rgb]{0.68,0.00,0.00}{#1}}
\newcommand{\AnnotationTok}[1]{\textcolor[rgb]{0.37,0.37,0.37}{#1}}
\newcommand{\AttributeTok}[1]{\textcolor[rgb]{0.40,0.45,0.13}{#1}}
\newcommand{\BaseNTok}[1]{\textcolor[rgb]{0.68,0.00,0.00}{#1}}
\newcommand{\BuiltInTok}[1]{\textcolor[rgb]{0.00,0.23,0.31}{#1}}
\newcommand{\CharTok}[1]{\textcolor[rgb]{0.13,0.47,0.30}{#1}}
\newcommand{\CommentTok}[1]{\textcolor[rgb]{0.37,0.37,0.37}{#1}}
\newcommand{\CommentVarTok}[1]{\textcolor[rgb]{0.37,0.37,0.37}{\textit{#1}}}
\newcommand{\ConstantTok}[1]{\textcolor[rgb]{0.56,0.35,0.01}{#1}}
\newcommand{\ControlFlowTok}[1]{\textcolor[rgb]{0.00,0.23,0.31}{\textbf{#1}}}
\newcommand{\DataTypeTok}[1]{\textcolor[rgb]{0.68,0.00,0.00}{#1}}
\newcommand{\DecValTok}[1]{\textcolor[rgb]{0.68,0.00,0.00}{#1}}
\newcommand{\DocumentationTok}[1]{\textcolor[rgb]{0.37,0.37,0.37}{\textit{#1}}}
\newcommand{\ErrorTok}[1]{\textcolor[rgb]{0.68,0.00,0.00}{#1}}
\newcommand{\ExtensionTok}[1]{\textcolor[rgb]{0.00,0.23,0.31}{#1}}
\newcommand{\FloatTok}[1]{\textcolor[rgb]{0.68,0.00,0.00}{#1}}
\newcommand{\FunctionTok}[1]{\textcolor[rgb]{0.28,0.35,0.67}{#1}}
\newcommand{\ImportTok}[1]{\textcolor[rgb]{0.00,0.46,0.62}{#1}}
\newcommand{\InformationTok}[1]{\textcolor[rgb]{0.37,0.37,0.37}{#1}}
\newcommand{\KeywordTok}[1]{\textcolor[rgb]{0.00,0.23,0.31}{\textbf{#1}}}
\newcommand{\NormalTok}[1]{\textcolor[rgb]{0.00,0.23,0.31}{#1}}
\newcommand{\OperatorTok}[1]{\textcolor[rgb]{0.37,0.37,0.37}{#1}}
\newcommand{\OtherTok}[1]{\textcolor[rgb]{0.00,0.23,0.31}{#1}}
\newcommand{\PreprocessorTok}[1]{\textcolor[rgb]{0.68,0.00,0.00}{#1}}
\newcommand{\RegionMarkerTok}[1]{\textcolor[rgb]{0.00,0.23,0.31}{#1}}
\newcommand{\SpecialCharTok}[1]{\textcolor[rgb]{0.37,0.37,0.37}{#1}}
\newcommand{\SpecialStringTok}[1]{\textcolor[rgb]{0.13,0.47,0.30}{#1}}
\newcommand{\StringTok}[1]{\textcolor[rgb]{0.13,0.47,0.30}{#1}}
\newcommand{\VariableTok}[1]{\textcolor[rgb]{0.07,0.07,0.07}{#1}}
\newcommand{\VerbatimStringTok}[1]{\textcolor[rgb]{0.13,0.47,0.30}{#1}}
\newcommand{\WarningTok}[1]{\textcolor[rgb]{0.37,0.37,0.37}{\textit{#1}}}

\providecommand{\tightlist}{%
  \setlength{\itemsep}{0pt}\setlength{\parskip}{0pt}}\usepackage{longtable,booktabs,array}
\usepackage{calc} % for calculating minipage widths
% Correct order of tables after \paragraph or \subparagraph
\usepackage{etoolbox}
\makeatletter
\patchcmd\longtable{\par}{\if@noskipsec\mbox{}\fi\par}{}{}
\makeatother
% Allow footnotes in longtable head/foot
\IfFileExists{footnotehyper.sty}{\usepackage{footnotehyper}}{\usepackage{footnote}}
\makesavenoteenv{longtable}
\usepackage{graphicx}
\makeatletter
\def\maxwidth{\ifdim\Gin@nat@width>\linewidth\linewidth\else\Gin@nat@width\fi}
\def\maxheight{\ifdim\Gin@nat@height>\textheight\textheight\else\Gin@nat@height\fi}
\makeatother
% Scale images if necessary, so that they will not overflow the page
% margins by default, and it is still possible to overwrite the defaults
% using explicit options in \includegraphics[width, height, ...]{}
\setkeys{Gin}{width=\maxwidth,height=\maxheight,keepaspectratio}
% Set default figure placement to htbp
\makeatletter
\def\fps@figure{htbp}
\makeatother

\usepackage{algorithm}
\usepackage{algpseudocode}
\usepackage[left=2cm,right=2cm,top = 2cm,bottom = 2cm]{geometry}
\usepackage[style=alphabetic]{biblatex}
\makeatletter
\@ifpackageloaded{tcolorbox}{}{\usepackage[skins,breakable]{tcolorbox}}
\@ifpackageloaded{fontawesome5}{}{\usepackage{fontawesome5}}
\definecolor{quarto-callout-color}{HTML}{909090}
\definecolor{quarto-callout-note-color}{HTML}{0758E5}
\definecolor{quarto-callout-important-color}{HTML}{CC1914}
\definecolor{quarto-callout-warning-color}{HTML}{EB9113}
\definecolor{quarto-callout-tip-color}{HTML}{00A047}
\definecolor{quarto-callout-caution-color}{HTML}{FC5300}
\definecolor{quarto-callout-color-frame}{HTML}{acacac}
\definecolor{quarto-callout-note-color-frame}{HTML}{4582ec}
\definecolor{quarto-callout-important-color-frame}{HTML}{d9534f}
\definecolor{quarto-callout-warning-color-frame}{HTML}{f0ad4e}
\definecolor{quarto-callout-tip-color-frame}{HTML}{02b875}
\definecolor{quarto-callout-caution-color-frame}{HTML}{fd7e14}
\makeatother
\makeatletter
\@ifpackageloaded{caption}{}{\usepackage{caption}}
\AtBeginDocument{%
\ifdefined\contentsname
  \renewcommand*\contentsname{Table of contents}
\else
  \newcommand\contentsname{Table of contents}
\fi
\ifdefined\listfigurename
  \renewcommand*\listfigurename{List of Figures}
\else
  \newcommand\listfigurename{List of Figures}
\fi
\ifdefined\listtablename
  \renewcommand*\listtablename{List of Tables}
\else
  \newcommand\listtablename{List of Tables}
\fi
\ifdefined\figurename
  \renewcommand*\figurename{Figure}
\else
  \newcommand\figurename{Figure}
\fi
\ifdefined\tablename
  \renewcommand*\tablename{Table}
\else
  \newcommand\tablename{Table}
\fi
}
\@ifpackageloaded{float}{}{\usepackage{float}}
\floatstyle{ruled}
\@ifundefined{c@chapter}{\newfloat{codelisting}{h}{lop}}{\newfloat{codelisting}{h}{lop}[chapter]}
\floatname{codelisting}{Listing}
\newcommand*\listoflistings{\listof{codelisting}{List of Listings}}
\makeatother
\makeatletter
\makeatother
\makeatletter
\@ifpackageloaded{caption}{}{\usepackage{caption}}
\@ifpackageloaded{subcaption}{}{\usepackage{subcaption}}
\makeatother
\makeatletter
\@ifpackageloaded{tikz}{}{\usepackage{tikz}}
\makeatother
        \newcommand*\circled[1]{\tikz[baseline=(char.base)]{
          \node[shape=circle,draw,inner sep=1pt] (char) {{\scriptsize#1}};}}  
                  
\ifLuaTeX
  \usepackage{selnolig}  % disable illegal ligatures
\fi
\usepackage[]{biblatex}
\addbibresource{refs.bib}
\usepackage{bookmark}

\IfFileExists{xurl.sty}{\usepackage{xurl}}{} % add URL line breaks if available
\urlstyle{same} % disable monospaced font for URLs
\hypersetup{
  pdftitle={Report on SINDy progress},
  pdfauthor={Gage Bonner and Michael Castellucci},
  colorlinks=true,
  linkcolor={blue},
  filecolor={Maroon},
  citecolor={Blue},
  urlcolor={Blue},
  pdfcreator={LaTeX via pandoc}}

\title{Report on SINDy progress}
\author{Gage Bonner and Michael Castellucci}
\date{}

\begin{document}
\maketitle

\renewcommand*\contentsname{Table of contents}
{
\hypersetup{linkcolor=}
\setcounter{tocdepth}{3}
\tableofcontents
}
\begin{center}\rule{0.5\linewidth}{0.5pt}\end{center}

\section{Summary of work}\label{summary-of-work}

In this project, we investigated the application of \emph{data
discovery} algorithms to learn the governing equations for a dynamical
system from numerical realizations of their trajectories. We apply the
sparse identification of nonlinear dynamics (``SINDy'') as originally
proposed in \cite{brunton2016discovering} as well as a number of its
extensions. The core idea behind these algorithms is that many dynamical
systems in science are represented by differential equation systems
\(\dot{x} = f(x)\) where \(f(x)\) often a linear combination of a small
number of elementary functions. By contrast, the trajectories \(x(t)\)
may be extremely complicated and ill-suited to direct fitting. Consider
the classic Lorenz system \begin{subequations} \label{eq:lorenz-def}
\begin{align} 
    \dot{x} &= 10 (y - x), \\
    \dot{y} &= x (28 - z) - y, \\ 
    \dot{z} &= x y - (8 / 3) z .
\end{align}
\end{subequations} Trajectories \(X(t) = (x(t), y(t), z(t))\) of this
system are chaotic, making it difficult or impossible to propose a
function that fits \(X(t)\) directly. On the other hand, \(\dot{X}\) is
a low order polynomial function of \(X(t)\) and hence should be much
more tractable in principle. At the highest level, we therefore have the
following problem: given numerical trajectories
\(\{x(t_1), x(t_2), \dots \}\), compute numerically
\(\{\dot{x}(t_1), \dot{x}(t_2), \dots \}\) and attempt to find a
parsimonious (equivalently: \emph{sparse}) combination of simple
functions of the \(x(t)\) that faithfully represents it.

This report contains embedded code from the
\href{https://julialang.org/}{Julia} programming language.

\section{Core algorithms}\label{core-algorithms}

\subsection{Sparse representations}\label{sparse-representations}

Suppose that several values of a function of
\(f : \mathbb{R} \to \mathbb{R}\) are observed,
\(\{f(x_1), f(x_2), \dots \}\). A \emph{sparse representation} seeks to
represent \(f\) as a linear combination of elementary functions of \(x\)
that contains as few terms as possible while still faithfully
representing the behavior of \(f\). As a first example, take
\(f(x) = x + \sin(x)\). We will add a small noise term to simulate real
data. This is shown in Figure \ref{fig:f-1}.

\begin{Shaded}
\begin{Highlighting}[]
\FunctionTok{f\_simple\_1}\NormalTok{(x) }\OperatorTok{=}\NormalTok{ x }\OperatorTok{+} \FunctionTok{sin}\NormalTok{(x)}
\NormalTok{x\_points\_1 }\OperatorTok{=} \FunctionTok{range}\NormalTok{(}\OperatorTok{{-}}\FloatTok{5}\NormalTok{, }\FloatTok{5}\NormalTok{, length }\OperatorTok{=} \FloatTok{20}\NormalTok{) }\OperatorTok{|\textgreater{}}\NormalTok{ collect}
\NormalTok{f\_points\_1 }\OperatorTok{=}\NormalTok{ [}\FunctionTok{f\_simple\_1}\NormalTok{(x) }\OperatorTok{+} \FloatTok{0.2}\FunctionTok{*rand}\NormalTok{() for x }\KeywordTok{in}\NormalTok{ x\_points\_1]}
\end{Highlighting}
\end{Shaded}

\begin{figure}
    \centering
    \includegraphics[width = 0.5\textwidth]{figures/f-1.png}
    \caption{A simple data set from a scalar function.}
    \label{fig:f-1}
\end{figure}

Assuming now that we are given this data with no knowledge of the
underlying mechanics, how could this function be represented? Due to the
oscillatory nature of the function, we might assume that it may be some
linear combination of simple polynomials and sinusoids. Provided this
sort of ``expert knowledge'', we could propose a \emph{library} of
functions \(L(x)\) that constitute the set of all possible functions
that we want to include in our model. In our case, we will take our
library to be the vector \begin{equation}
L(x) = (1, x, x^2, x^3, \sin x, \cos x).
\end{equation} The problem is therefore to find a vector
\(\xi \in \mathbb{R}^7\) such that \(f(x) \approx \xi \cdot L(x)\). To
do this, we will construct an optimization problem whose solution is
\(\xi\) that takes all of our observational data into account. The
\emph{library data} \(\Theta\) is given by \begin{equation}
\Theta(x) 
= 
\begin{pmatrix}
L_1(x_1) & L_2(x_1) & \cdots & L_7(x_1) \\ 
L_1(x_2) & L_2(x_2) & \cdots & L_7(x_2) \\
\vdots   & \vdots   & \ddots & \vdots   \\
L_1(x_N) & L_2(x_N) & \cdots & L_7(x_N) 
\end{pmatrix}
=
\begin{pmatrix}
1 & x_1 & \cdots & \cos(x_1) \\ 
1 & x_2 & \cdots & \cos(x_2) \\
\vdots   & \vdots   & \ddots & \vdots   \\
1 & x_N & \cdots & \cos(x_N) 
\end{pmatrix}.
\end{equation} Given our \emph{target data}
\(F = (f(x_1), f(x_2), \dots)\), we naturally have the following
optimization problem \begin{equation}
\xi^\star = \underset{\xi \in \mathbb{R}^7}{\text{argmin}} \lVert F - \Theta \xi \rVert_{2} ,
\end{equation} where \(\Vert\cdot\rVert_2\) indicates the \(L_2\) norm
and where our final representation is
\(f(x) \approx L(x) \cdot \xi^\star\). This kind of simple problem is
directly amenable to a linear least-squares solution.

\begin{Shaded}
\begin{Highlighting}[]
\NormalTok{library\_1 }\OperatorTok{=}\NormalTok{ [x }\OperatorTok{{-}\textgreater{}} \FloatTok{1.0}\NormalTok{, x }\OperatorTok{{-}\textgreater{}}\NormalTok{ x, x }\OperatorTok{{-}\textgreater{}}\NormalTok{ x}\OperatorTok{\^{}}\FloatTok{2}\NormalTok{, x }\OperatorTok{{-}\textgreater{}}\NormalTok{ x}\OperatorTok{\^{}}\FloatTok{3}\NormalTok{, x }\OperatorTok{{-}\textgreater{}} \FunctionTok{sin}\NormalTok{(x), x }\OperatorTok{{-}\textgreater{}} \FunctionTok{cos}\NormalTok{(x)]}
\NormalTok{library\_names\_1 }\OperatorTok{=}\NormalTok{ [}\StringTok{"1"}\NormalTok{, }\StringTok{"x"}\NormalTok{, }\StringTok{"x\^{}2"}\NormalTok{, }\StringTok{"x\^{}3"}\NormalTok{, }\StringTok{"sin(x)"}\NormalTok{, }\StringTok{"cos(x)"}\NormalTok{]}
\NormalTok{library\_data\_1 }\OperatorTok{=}\NormalTok{ [}\FunctionTok{f}\NormalTok{(t) for t }\KeywordTok{in}\NormalTok{ x\_points\_1, f }\KeywordTok{in}\NormalTok{ library\_1]}

\NormalTok{ls\_1 }\OperatorTok{=}\NormalTok{ MultivariateStats.}\FunctionTok{llsq}\NormalTok{(library\_data\_1, f\_points\_1, bias }\OperatorTok{=} \ConstantTok{false}\NormalTok{)}
\FunctionTok{f\_llsq\_1}\NormalTok{(x) }\OperatorTok{=} \FunctionTok{sum}\NormalTok{(ls\_1[i]}\OperatorTok{*}\NormalTok{library\_1[i](x) }\ControlFlowTok{for}\NormalTok{ i }\OperatorTok{=} \FloatTok{1}\OperatorTok{:}\FunctionTok{length}\NormalTok{(ls\_1))}
\end{Highlighting}
\end{Shaded}

\begin{equation} \label{eq:f-simple-1-sparse}   f_{\text{llsq}}(x) = 0.124 \cdot 1 + 1.02 \cdot x -0.00307 \cdot x^{2} -0.001 \cdot x^{3} + 0.992 \cdot \sin\left( x \right) -0.0102 \cdot \cos\left( x \right) \end{equation}

We can see that the fit produced is indeed quite good as in Figure
\ref{fig:f-simple-llsq}.

\begin{figure}
    \centering
    \includegraphics[width = 0.5\textwidth]{figures/f-1-llsq.png}
    \caption{The linear least squares result.}
    \label{fig:f-simple-llsq}
\end{figure}

\clearpage

\begin{Shaded}
\begin{Highlighting}[]
\CommentTok{\# let}
\NormalTok{times }\OperatorTok{=}\NormalTok{ x\_points\_1 }\OperatorTok{|\textgreater{}}\NormalTok{ collect}
\NormalTok{target\_data }\OperatorTok{=}\NormalTok{ f\_points\_1}

\NormalTok{library }\OperatorTok{=}\NormalTok{ [x }\OperatorTok{{-}\textgreater{}} \FloatTok{1.0}\NormalTok{, x }\OperatorTok{{-}\textgreater{}}\NormalTok{ x, x }\OperatorTok{{-}\textgreater{}}\NormalTok{ x}\OperatorTok{\^{}}\FloatTok{2}\NormalTok{, x }\OperatorTok{{-}\textgreater{}}\NormalTok{ x}\OperatorTok{\^{}}\FloatTok{3}\NormalTok{, x }\OperatorTok{{-}\textgreater{}} \FunctionTok{sin}\NormalTok{(x), x }\OperatorTok{{-}\textgreater{}} \FunctionTok{cos}\NormalTok{(x)]}
\NormalTok{library\_names }\OperatorTok{=}\NormalTok{ [}\StringTok{"1"}\NormalTok{, }\StringTok{"x"}\NormalTok{, }\StringTok{"x\^{}2"}\NormalTok{, }\StringTok{"x\^{}3"}\NormalTok{, }\StringTok{"sin(x)"}\NormalTok{, }\StringTok{"cos(x)"}\NormalTok{]}
\NormalTok{library\_data }\OperatorTok{=}\NormalTok{ [}\FunctionTok{f}\NormalTok{(t) for t }\KeywordTok{in}\NormalTok{ times, f }\KeywordTok{in}\NormalTok{ library]}

\NormalTok{λ\_sparse }\OperatorTok{=} \FloatTok{0.5}
\NormalTok{λ\_ridge }\OperatorTok{=} \FloatTok{0.1}

\NormalTok{ls }\OperatorTok{=}\NormalTok{ MultivariateStats.}\FunctionTok{llsq}\NormalTok{(library\_data, target\_data, bias }\OperatorTok{=} \ConstantTok{false}\NormalTok{)}
\NormalTok{pp\_ls }\OperatorTok{=} \FunctionTok{pretty\_print}\NormalTok{(ls, library\_names)}

\NormalTok{rr }\OperatorTok{=}\NormalTok{ MultivariateStats.}\FunctionTok{ridge}\NormalTok{(library\_data, target\_data, λ\_ridge, bias }\OperatorTok{=} \ConstantTok{false}\NormalTok{)}
\NormalTok{pp\_rr }\OperatorTok{=} \FunctionTok{pretty\_print}\NormalTok{(rr, library\_names)}

\NormalTok{sr }\OperatorTok{=} \FunctionTok{sparse\_representation}\NormalTok{(times, target\_data, library\_data, λ\_sparse }\OperatorTok{=}\NormalTok{ λ\_sparse, λ\_ridge }\OperatorTok{=}\NormalTok{ λ\_ridge)}
\NormalTok{pp\_sr }\OperatorTok{=} \FunctionTok{pretty\_print}\NormalTok{(sr, library\_names)}

\NormalTok{md}\StringTok{"""}
\SpecialCharTok{\textbackslash{}b}\StringTok{egin\{subequations\} \textbackslash{}label\{eq:f{-}simple{-}1{-}sparse\}}
\SpecialCharTok{\textbackslash{}b}\StringTok{egin\{align\} }
\StringTok{  \textbackslash{}mathrm\{llsq\} \&= }\SpecialCharTok{$}\NormalTok{(}\FunctionTok{latexify}\NormalTok{(pp\_ls, env }\OperatorTok{=} \OperatorTok{:}\NormalTok{raw, fmt  }\OperatorTok{=} \FunctionTok{FancyNumberFormatter}\NormalTok{(}\FloatTok{3}\NormalTok{)))}\StringTok{ }\SpecialCharTok{\textbackslash{}\textbackslash{}\textbackslash{}\textbackslash{}}
\StringTok{  \textbackslash{}mathrm\{ridge\} \&= }\SpecialCharTok{$}\NormalTok{(}\FunctionTok{latexify}\NormalTok{(pp\_rr, env }\OperatorTok{=} \OperatorTok{:}\NormalTok{raw, fmt  }\OperatorTok{=} \FunctionTok{FancyNumberFormatter}\NormalTok{(}\FloatTok{3}\NormalTok{)))}\StringTok{ }\SpecialCharTok{\textbackslash{}\textbackslash{}\textbackslash{}\textbackslash{}}
\StringTok{  \textbackslash{}mathrm\{sparse\} \&= }\SpecialCharTok{$}\NormalTok{(}\FunctionTok{latexify}\NormalTok{(pp\_sr, env }\OperatorTok{=} \OperatorTok{:}\NormalTok{raw, fmt  }\OperatorTok{=} \FunctionTok{FancyNumberFormatter}\NormalTok{(}\FloatTok{3}\NormalTok{)))}
\SpecialCharTok{\textbackslash{}e}\StringTok{nd\{align\}}
\SpecialCharTok{\textbackslash{}e}\StringTok{nd\{subequations\}}
\StringTok{"""}
\CommentTok{\# end}
\end{Highlighting}
\end{Shaded}

\begin{subequations} \label{eq:f-simple-1-sparse} \begin{align}    \mathrm{llsq} &= 0.124 \cdot 1 + 1.02 \cdot x -0.00307 \cdot x^{2} -0.001 \cdot x^{3} + 0.992 \cdot \sin\left( x \right) -0.0102 \cdot \cos\left( x \right) \\
  \mathrm{ridge} &= 0.123 \cdot 1 + 1.02 \cdot x -0.00298 \cdot x^{2} -0.00131 \cdot x^{3} + 0.974 \cdot \sin\left( x \right) -0.00995 \cdot \cos\left( x \right) \\
  \mathrm{sparse} &= 1 \cdot x + 1.01 \cdot \sin\left( x \right) \end{align} \end{subequations}

As a first example, take \(f(x) = x + \sin(x) + \cos(x)^2\). We will add
a small noise term to simulate real data.

\begin{Shaded}
\begin{Highlighting}[]
\FunctionTok{f\_simple}\NormalTok{(x) }\OperatorTok{=}\NormalTok{ x }\OperatorTok{+} \FunctionTok{sin}\NormalTok{(x) }\OperatorTok{+} \FunctionTok{cos}\NormalTok{(x)}\OperatorTok{\^{}}\FloatTok{2}
\NormalTok{x\_points }\OperatorTok{=} \FunctionTok{range}\NormalTok{(}\OperatorTok{{-}}\FloatTok{5}\NormalTok{, }\FloatTok{5}\NormalTok{, length }\OperatorTok{=} \FloatTok{100}\NormalTok{)}
\NormalTok{f\_points }\OperatorTok{=}\NormalTok{ [}\FunctionTok{f\_simple}\NormalTok{(x) }\OperatorTok{+} \FloatTok{0.2}\FunctionTok{*rand}\NormalTok{() for x }\KeywordTok{in}\NormalTok{ x\_points]}
\end{Highlighting}
\end{Shaded}

\begin{itemize}
\tightlist
\item
  Library functions
\item
  need ridge regression since not positive definite
\end{itemize}

\begin{figure}
    \centering
    \includegraphics[width = 0.5\textwidth]{figures/test.png}
    \caption{test}
    \label{fig:test}
\end{figure}

\begin{Shaded}
\begin{Highlighting}[]
\KeywordTok{let}
\NormalTok{times }\OperatorTok{=}\NormalTok{ x\_points }\OperatorTok{|\textgreater{}}\NormalTok{ collect}
\NormalTok{target\_data }\OperatorTok{=}\NormalTok{ f\_points}

\NormalTok{library }\OperatorTok{=}\NormalTok{ [x }\OperatorTok{{-}\textgreater{}} \FloatTok{1.0}\NormalTok{, x }\OperatorTok{{-}\textgreater{}}\NormalTok{ x, x }\OperatorTok{{-}\textgreater{}}\NormalTok{ x}\OperatorTok{\^{}}\FloatTok{2}\NormalTok{, x }\OperatorTok{{-}\textgreater{}}\NormalTok{ x}\OperatorTok{\^{}}\FloatTok{3}\NormalTok{, x }\OperatorTok{{-}\textgreater{}} \FunctionTok{sin}\NormalTok{(x), x }\OperatorTok{{-}\textgreater{}} \FunctionTok{cos}\NormalTok{(x)]}
\NormalTok{library\_names }\OperatorTok{=}\NormalTok{ [}\StringTok{"1"}\NormalTok{, }\StringTok{"x"}\NormalTok{, }\StringTok{"x\^{}2"}\NormalTok{, }\StringTok{"x\^{}3"}\NormalTok{, }\StringTok{"sin(x)"}\NormalTok{, }\StringTok{"cos(x)"}\NormalTok{]}
\NormalTok{library\_data }\OperatorTok{=}\NormalTok{ [}\FunctionTok{f}\NormalTok{(t) for t }\KeywordTok{in}\NormalTok{ times, f }\KeywordTok{in}\NormalTok{ library]}

\NormalTok{λ\_sparse }\OperatorTok{=} \FloatTok{0.5}
\NormalTok{λ\_ridge }\OperatorTok{=} \FloatTok{0.1}

\NormalTok{rr }\OperatorTok{=}\NormalTok{ MultivariateStats.}\FunctionTok{ridge}\NormalTok{(library\_data, target\_data, λ\_ridge, bias }\OperatorTok{=} \ConstantTok{false}\NormalTok{)}
\CommentTok{\# @info pretty\_print(rr, library\_names)}

\NormalTok{sr }\OperatorTok{=} \FunctionTok{sparse\_representation}\NormalTok{(times, target\_data, library\_data, λ\_sparse }\OperatorTok{=}\NormalTok{ λ\_sparse, λ\_ridge }\OperatorTok{=}\NormalTok{ λ\_ridge)}
\NormalTok{pp }\OperatorTok{=} \FunctionTok{pretty\_print}\NormalTok{(sr, library\_names)}

\NormalTok{md}\StringTok{"""}
\SpecialCharTok{\textbackslash{}b}\StringTok{egin\{equation\}}
\SpecialCharTok{$}\NormalTok{(}\FunctionTok{latexify}\NormalTok{(pp, env }\OperatorTok{=} \OperatorTok{:}\NormalTok{raw))}
\SpecialCharTok{\textbackslash{}e}\StringTok{nd\{equation\}}
\StringTok{"""}
\KeywordTok{end}
\end{Highlighting}
\end{Shaded}

\begin{equation} 0.566 \cdot 1 + 1.0 \cdot x + 0.998 \cdot \sin\left( x \right) \end{equation}

\subsection{SINDy}\label{sindy}

\subsection{E-SINDy}\label{e-sindy}

\section{Julia Implementation}\label{julia-implementation}

\section{Results}\label{results}

\subsection{Direct sparse
representations}\label{direct-sparse-representations}

\subsection{Fluid}\label{fluid}

\subsection{Slow}\label{slow}

\subsection{Full}\label{full}

\section{Conclusions}\label{conclusions}

\begin{Shaded}
\begin{Highlighting}[]
\FloatTok{1} \OperatorTok{+} \FloatTok{2}
\end{Highlighting}
\end{Shaded}

\begin{verbatim}
3
\end{verbatim}

\begin{Shaded}
\begin{Highlighting}[]
\FloatTok{1} \OperatorTok{+} \FloatTok{2}
\end{Highlighting}
\end{Shaded}

\begin{verbatim}
3
\end{verbatim}

Blah blah \(sdfadf\) \begin{equation}
F = ma
\end{equation}

\begin{tcolorbox}[enhanced jigsaw, breakable, coltitle=black, toptitle=1mm, colbacktitle=quarto-callout-tip-color!10!white, bottomrule=.15mm, titlerule=0mm, opacityback=0, leftrule=.75mm, title={Algorithm 1 (stock)}, bottomtitle=1mm, arc=.35mm, colback=white, left=2mm, rightrule=.15mm, colframe=quarto-callout-tip-color-frame, toprule=.15mm, opacitybacktitle=0.6]

\phantomsection\label{annotated-cell-10}%
\begin{Shaded}
\begin{Highlighting}[]
\NormalTok{penguins }\OperatorTok{=} \FloatTok{1} \hspace*{\fill}\NormalTok{\circled{1}}
\NormalTok{penguins}\OperatorTok{*}\FloatTok{2} \hspace*{\fill}\NormalTok{\circled{2}}
\end{Highlighting}
\end{Shaded}

\begin{description}
\tightlist
\item[\circled{1}]
Take \texttt{penguins}, and then,
\item[\circled{2}]
add new columns for the bill ratio and bill area\ldots{}
\end{description}

\begin{verbatim}
2
\end{verbatim}

\end{tcolorbox}

\begin{algorithm}
\caption{An algorithm with caption}\label{alg:cap}
\begin{algorithmic}
\Require $n \geq 0$
\Ensure $y = x^n$
\State $y \gets 1$
\State $X \gets x$
\State $N \gets n$
\While{$N \neq 0$}
\If{$N$ is even}
    \State $X \gets X \times X$
    \State $N \gets \frac{N}{2}$  \Comment{This is a comment}
\ElsIf{$N$ is odd}
    \State $y \gets y \times X$
    \State $N \gets N - 1$
\EndIf
\EndWhile
\end{algorithmic}
\end{algorithm}


\printbibliography


\end{document}
